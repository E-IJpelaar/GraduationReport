\section{Conclusion}

In this thesis, model-based feedback control applied to a single-link soft robotic manipulator is investigated. The performance of the control strategy is analyzed utilizing a simulation model and experiments. This developed simulation model incorporates the dynamics of the manipulator and air pumps. The experimental setup enabled analyzing the controller performance for real-time set-point regulation and reference tracking. This study focuses on a two-bellow soft robot actuated by air pressure. Previous work \cite{berkers} conducted on the same soft robot limited itself to linear control. In that sense, this work approached the control problem with a changed perspective. 


First, a kinematic description of the soft robot is derived based on the Cosserat beam theory. This resulted in a set of partial differential equations describing the robot configuration space. Employing a Galerkin reduction allowed to express the forward kinematics by an ordinary differential equation. The velocity kinematics are used to construct the systems space-variant Jacobian matrix. Based on Euler-Lagrange equations a non-linear dynamic model for the manipulator is derived. To capture the overall dynamics the pump dynamics as first-order systems. Combining the models resulted in an overall description of the system dynamics. 

A parameter study is conducted to obtain the stiffness properties of the soft robot. A FEM model of the soft robot is utilized to study pressure and force-induced deformation. These results are used to extract the non-linear elongation and curvature stiffness. A first-order pressure model is derived by analyzing the system's response to a sinusoidal Volt input signal. Damping properties could not be experimentally acquired.

Then a control architecture for controlling the system is designed. This architecture comprises a model-based feedback controller inspired by the soft robots space-variant Jacobian matrix. This Jacobian controller determines the pressure reference necessary to reach a given set-point in the xy-plane. The model-based controller is accompanied by a PI controller to regulate the bellow pressures. 

The control architecture is verified in simulation and experimentally. The derived dynamic model is used to tune the Jacobian controller and pressure controller in simulation. To this end, the response to a position step input is considered. For the simulation, this resulted in settling times of 16 seconds in both x and y-direction. Then the tracking of an ellipsoid reference path is studied in simulation. The noise-free simulation showed desirable tracking characteristics, with RMS errors of $0.1489 \hspace{2pt} mm$ and $0.0899 \hspace{2pt} mm$ in x and y-direction, respectively. The tracking error is caused by delays in the system. These are actuation delays caused by the pressure dynamics and control delays due to low-pass filtering of the control input. The controller is also implemented on the experimental setup. Additional filters are needed to improve sensor readings. Again, a step response is used to tune the system. The observed settling times are 12 and 8 seconds x and y-direction, respectively. The best achieved RMS errors for these step responses in x and y-direction are $0.26034 \hspace{2pt} mm$ and $0.2644 \hspace{2pt} mm$, respectively. The tracking performance is evaluated using the same reference path as in the simulation. During tracking, the achieved RMS errors are $0.7926 \hspace{2pt} mm$ and $0.4044 \hspace{2pt}  mm$ in x and y-direction, respectively. 

Conclusively, the achieved results in set-point regulation and reference tracking are satisfying. The obtained results during tracking show significant improvement with respect to \cite{berkers}. The derived dynamic model for the manipulator has shortcomings in capturing the soft robot's dynamics. However, the overall system model, which includes the pressure dynamics, compares well to the physical system. 

\section{Recommendation}

Several recommendations can be given to improve results. The following section lists improvements on the topics: modelling, setup and control design.

Firstly, improvements can be made concerning modelling. In this work, the derived dynamic model is based on a first-order Galerkin reduction, essentially reducing the model to the constant curvature approach. Although, constant curvature is a fairly accurate description of the soft robot's kinematics. The versatile Cosserat beam model can be exploited such that higher-order dynamics can also be incorporated. A higher-order reduction yields an improved representation of the theoretical infinite degree-of-freedom soft robotic system. Furthermore, it would allow studying items such as under-actuation and higher-order dynamics. Regarding this study, the mass approximation can be further improved. Although the stiffnesses seemed to be well captured, the low mass and inertia approximation led to incorrect high eigenfrequencies. Additional model improvements could be the inclusion of Coriolis effects and gravitation effects. Coriolis effects become increasingly important when the system is tuned to perform faster reference trajectories. Gravity effects are relatively simple to include in the model, and give a completer view of the actual forces acting on the system. The last model concern focuses on pressure modelling. A first-order model represents the inflation of the soft robot well. However, deflation of the actuator is neglected. Model-wise this can also be included by a first-order linear deflation model. 

In terms of setup, several improvements can be made. First of all the system can not be actively deflated. Therefore deflation rates are an important limitation when it comes to high-speed reference changes. To overcome this limitation a controllable valve could be added to the setup to deflate the bellows quickly. Also, the choice can be made to introduce vacuum pumps to deflate the soft robot's bellows. When it comes to air pumps a major improvement can also be the introduction of pressurized vessels. Contrary to diaphragm pumps, pressurized vessels allow for a constant inflow of air. Therefore it would remove the valve dynamics entirely. Furthermore, it would allow for quicker inflation which can improve performance. Additionally, air pumps saturation limits do not need to be taken into account anymore. Other improvements can be made for data acquisition. A major limitation is the sampling frequency due to serial communication between the Raspberry PI and Arduino. Desirably, all control actions and sensor readings would be done by the Raspberry. This would increase the sampling frequency and allows for real-time evaluation of the results. The current vision system has a 60 Hz update rate, which is a considerable limitation when demanding higher performance. In this study, a low-pass filter is used to give a better position estimate. However, this does add additional delays in the system. Therefore a higher update rate and higher resolution can be beneficial. A final improvement on the setup is the LED marker, in this study a yellow LED is used. However, the hue value is close to the soft robot's hue. Therefore, vision system tuning is cumbersome. It would be better to replace this LED with another coloured LED. 


Lastly, there is a lot of performance to be gained with improving the control architecture. In this study, only a feedback controller is considered. In soft robotics, feedback control conflicts with the intrinsic properties of soft robotics. Feedback control namely introduces stiffness, whilst the soft robot is designed to be flexible. To reduce this effect and enhance performance, feedforward can be introduced. In the current control architecture, the pressure reference increases slowly as a result of the integrator action. Since stiffness models are available, the addition of stiffness compensation in the Jacobian controller will ensure a faster increase of pressure set point. Together with a feedforward model on the pumps, this will allow for faster reference tracking. Another improvement would be the inclusion of gravity compensation. Of course, control methods that involve a more model-based approach can be implemented. Promising methods, especially for under-actuated systems, include computed-torque control, (partial) feedback linearization and passivity-based control.  
