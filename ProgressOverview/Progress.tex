\section{Detailed Planning Regarding Experiments}


In order to effectively use the limited lab time, I would like to propose the following planning. Is consists of two phases, Preparation phase and Experimental phase. First mentioned includes every step that needs to be taken in order to have the set-up ready. The latter is the actual experiments with the proposed controller.


\subsection{Preparation Phase}


\textbf{Rotation measurement - 25/1 - 29/1}

In this single workweek it should be feasible to obtain the rotation data from the IMU. The complementary filter has already been installed, so it should be easy to retrieve angle information. In this week I would also like to obtain a graph with $\theta$ as function of time. This will be of use in finding model parameters regarding damping. But will also be a sort of reference point as to how the actuator behaves in a free oscillation. Out of plane motion will of course be present, but I think it will still be a useful way to see the dynamic behaviour in general.


\textbf{Pump control - 2/2 - 12/2}

Two work weeks to get the pressure sensor working and make a PI controller to regulate pressure levels. I am convinced that 2 weeks will be more than enough to develop a decent feedback controller. For this test the actuator will be connected to the set-up. A pump connected to a single bellow will be enough to develop a controller. Once this single controller is designed it will be duplicated for the other pump as well. It is assumed that both pumps have the same dynamics and characteristics. The obtained PI controller can always be slightly adapted once the set-up is complete as this is just a matter of playing with P and I gains.


\textbf{Vision system - 15/2 - 22/2}

The goal is to perform marker tracking with the Pixy2 vision system. I have never worked with something like this, so I don't know how hard it will be. However, I have found enough videos and literature online. Therefore I think it will be feasible to get Pixy2 working within one week. The marker will be firmly placed at the tip of the actuator. Together with the pressure controller it should be possible to measure x-y position in just one week.

\subsection{Experimental Phase}


\textbf{Quasi-static model validation - 22/2 - 5/3}

In these 2 weeks I would like to compare the actual actuator with my dynamic model quasi-statically. I think a dynamic validation will be hard since we cannot deflate the system instantly in order to create a free oscillation. In this experiment I want to pressurize the bellows for various combinations and check for elongation and rotation. I want to compare those quasi-static data in order to say something about the determined stiffness properties. Also, this validation will be important before designing a model-based controller. If the model is to far of reality it will probably not be useful to include in the controller. A table that I am looking to complete in these weeks look something like this:




\begin{table}[H]
    \centering
    \begin{tabular}{|l|l|l|l|l|l|}
    \hline
    \multicolumn{2}{|l|}{Elongation}   & \multicolumn{2}{l|}{Model $(x-y)$ mm}      &  \multicolumn{2}{l|}{Actuator }  \\  \cline{1-6}
      $p_1$ kPa & $p_2$ kPa  &      Linear $K$ & Non-linear $K$         &         (x-y) &  Angle $\theta$       \\ \hline
         10     &    10      &              &                           &               &      \\
         20     &    20      &              &                           &               &      \\
         30     &    30      &              &                           &               &      \\
         40     &    40      &              &                           &               &      \\
         50     &    50      &              &                           &               &      \\
         60     &    60      &              &                           &               &      \\         
         70     &    70      &              &                           &               &      \\
         80     &    80      &              &                           &               &      \\
    \hline
\end{tabular}
    \caption{Elongation validation. The model will give $\theta = 0$ when pressurizing both bellows. For the actuator this can differ.}
    \label{tab:elongat}
\end{table}



\begin{table}[H]
    \centering
    \begin{tabular}{|l|l|l|l|l|l|}
    \hline
    \multicolumn{2}{|l|}{Curvature}   & \multicolumn{2}{l|}{Model $\theta$}      &  \multicolumn{2}{l|}{Actuator }  \\  \cline{1-6}
      $p_1$ kPa & $p_2$ kPa  &      Linear $K$ & Non-linear $K$        &         (x-y) &  Angle $\theta$       \\ \hline
         0     &    10      &              &                           &               &      \\
         0     &    20      &              &                           &               &      \\
         0     &    30      &              &                           &               &      \\
         0     &    40      &              &                           &               &      \\
         0     &    50      &              &                           &               &      \\
         0     &    60      &              &                           &               &      \\         
         0     &    70      &              &                           &               &      \\
         0     &    80      &              &                           &               &      \\
    \hline
\end{tabular}
    \caption{Curvature validation.}
    \label{tab:curv}
\end{table}


\textbf{Quasi-static model validation - 5/3 - 22/3}

These three weeks will be dedicated to controller testing. Hopefully, the model will capture the dynamic behaviour well enough in order to be useful in the control strategy.