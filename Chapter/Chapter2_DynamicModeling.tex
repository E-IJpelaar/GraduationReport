This chapter explains the dynamic model derivation of the soft robotic manipulator. Forward kinematics describe the robots configuration for a given set of initial conditions. These derived forward kinematics allow to derive a dynamic model. 


%%%%%%%%%%%%%%%%%%%%%%%%%%%%
%%%%%%%%%%%%%%%%%%%%%%%%%%%%

\section{Kinematics}

The soft actuator studied in this thesis is shown in Figure \ref{fig2:kinematicschematic}. It consists of two bellows placed in parallel. The outside of the bellows are connected, forming the centre line of the actuator. Furthermore, the bellows are connected by a flange at the top and bottom. The soft robot is pneumatically actuated using air pumps. Each bellow can be inflated individually via a hole at one end of the actuator. This flange will be fixed to the ground. The other end is closed, allowing to pressurize the bellows. The actuators geometry and material choice allows the bellows to extend when pressurized. Since each bellow can be inflated independently, the entire actuator can increase its length and change orientation. The actuator is actuated in a single plane, hence the name planar soft actuator. Undesired out of plane motions are deemed negligibly small.

  \begin{minipage}{\linewidth}
      \centering
      \begin{minipage}{0.45\linewidth}
        \begin{figure}[H]
        \centering
            \includegraphics[width=\textwidth]{Figures/Chapter2/schematickinematic.png}
            \caption{Planar soft actuator with the backbone curve $g(\sigma,t)$.}
            \label{fig2:kinematicschematic}
        \end{figure}
      \end{minipage}
      \hspace{0.05\linewidth}
      \begin{minipage}{0.45\linewidth}
          \begin{figure}[H]
              \includegraphics[width=\linewidth]{Figures/Chapter2/ccapproach.png}
              \caption{Schematic drawing of the constant curvature \cite{ccapproach}.}
              \label{fig2:ccapproach}
          \end{figure}
      \end{minipage}
  \end{minipage}

\add{Figure should be adapted with sorotoki one, this looks nicer.}\\


A kinematic description of the actuator is necessary to describe the position in 3D space. Before deriving our kinematic model, we will be briefly discuss constant curvature modeling. Which is a widely used model to describe simplified kinematics of soft actuators \cite{ccapproach},\cite{berkers},\cite{Falkenhahn2015}. A schematic drawing of the constant curvature modelling approach is shown in Figure \ref{fig2:ccapproach}. Although we will not be using this exact methodology, it is important to understand the basics. The constant curvature describes the position of the actuator by three coordinates. Parameter $l$ is the curved length of the actuator measured from the fixed bottom to the tip. Coordinate $\kappa$ expresses the curvature of the actuator. It is assumed that the deformed actuator describes a perfect arc, hence radius $r$ is equal to $\frac{l}{k}$. This allows to write the orientation of the actuator's tip as $\theta = l\kappa$. Lastly, parameter $\phi$ describes the rotation of the actuator with respect to the ground. 

As mentioned, our modelling approach distinguishes itself from the constant curvature model. To describe the kinematics of the soft actuator, a Cosserat beam model is used \cite{Boyer2019}. This beam model can be thought of as a continuous 1-dimensional curve representing the robot's backbone. This backbone is represented in Figure (\ref{fig2:kinematicschematic}) as the black curve. This curve describes the configuration of the soft robot as a function of space and time. To this end, a spatial coordinate  $\sigma \in \mathbb{X}$ within bounded domain $\mathbb{X} \in [0,l] \subset \mathbb{R}$ is introduced. Furthermore, a temporal coordinate $t \in  \mathbb{T}$ with $\mathbb{T} \subseteq \mathbb{R}$ is defined. This allows to describe position $p(\sigma,t) \in \mathbb{R}^3$ and rotation $R(\sigma,t) \in \mathbb{SO}(3)$ for any point $\sigma$ and time instance $t$ among the backbone of the soft manipulator by \cite{Caasenbrood2020},


\begin{equation}
    g(\sigma,t) = \begin{pmatrix}  R(\sigma,t) & p(\sigma,t) \\ 0_3^\top & 1 \end{pmatrix} \in \mathbb{SE}(3),
    \label{eq2:g}
\end{equation}

where $\mathbb{SE}(3)$ is the Lie group of rigid body transformations in $\mathbb{R}^3$ \cite{Sola2018}. Essentially, function $g(\sigma,t)$ expresses stretch-strain in a local frame at $\sigma$. The forward kinematic problem can be found by differentiating (\ref{eq2:g}) with respect to position. This results in the following partial differential equation (PDE), 


\begin{equation}
    \frac{\partial g}{\partial \sigma} = g \hat{\xi} \hspace{10pt} \text{with} \hspace{10pt}  \hat{\xi} = \begin{pmatrix} K_\times & E \\ 0_3^\top & 0 \end{pmatrix} \in  \mathfrak{se}(3)
    \label{eq2:dgdsigma}
\end{equation}

where $\hat{\xi}$ is the space-twist field. Here, $K_\times$ is a skew-symmetric matrix expressing curvature-torsion strain, and $E = [\epsilon_1,\epsilon_2,\epsilon_3]^\top$ a vector containing stretch-shear strain. The entries of $E$ represent  degrees-of-freedom that allow elongation in all three directions. Likewise, the skew-symmetric matrix $K_{\times}$ holds three rotational degrees of freedom.  From this skew-symmetric matrix, vector $K = [\kappa_1,\kappa_2,\kappa_3]^\top$ can be derived \cite{Sola2018}. For the planar robot, it shall be clear that there is one elongation and one rotational degree-of-freedom. In order to include these degrees-of-freedom in the kinematic model, $\epsilon_1$ and $\kappa_2$ will have value 1. All other degrees-of-freedom will take value 0. Defining $\xi(\sigma,t)$ as $[K \hspace{2pt} E]^\top$ results in vector $[0,1,0,1,0,0]^\top$. Throughout this work we will use "elongation" or $\epsilon$ to address strain $\epsilon_1$. Likewise, "curvature" or $\kappa$ is used to refer to curvature $\kappa_2$.

To solve the PDE in (\ref{eq2:dgdsigma}), it transformed to an ordinary differential equation (ODE) by exploiting the Galerkin reduction method \cite{Galerkin}. Here, the infinite dimensional system is projected onto a subspace of finite dimension that contains basis elements of the expected solution. By reducing the dimensionality of the system, higher order dynamics are not captured in the model and thus robustness should be taken into account. To transform PDE (\ref{eq2:dgdsigma}), the components of the strain field $\xi(\sigma,t)$ are approximated using a finite amount of shape functions as,

\begin{equation}
    \xi_i(\sigma,t) \approx \sum_{i=0}^N \varphi_i(\sigma)q_i(t) + \xi_{i,0}(\sigma), \hspace{20pt} \forall \sigma \in \mathbb{X}, t \in \mathbb{T},
\end{equation}

in which, $\xi_{i,0}$ corresponds to the undeformed configuration of the robotic manipulator. Additionally, $\{\varphi_i\}_{i \in \mathbb{N}}$ is a set of basis shape functions and $q(t)$ are modal coefficients. These modal coefficients can be thought of as generalized coordinates of the finite-dimensional system. To be clear on this reduction method, integer $N$ is the amount of shape functions used to estimate strain field $\xi(\sigma,t)$. Hence, index $i$ represents the order of the shape function. Multiple types shape function polynomials exist to approximate strain \cite{Galerkin}. The Chebyshev polynomial, Polynomial and Legendre polynomials are respectively given by,

\begin{equation}
    \varphi_{i,cheby}(\sigma) = \cos(i \sigma), \hspace{40pt} \varphi_{i,poly} = \sigma^i, \hspace{40pt} \varphi_{i,legend} = \frac{1}{2^i i!} \frac{d^i}{d\sigma^i}(\sigma^2-1)^i.
    \label{eq2:shapefunction}
\end{equation}


Each degree of shape-function gives the model a certain amount of flexibility. Therefore increasing the order of shape-functions allows the model to describe more complex robot configurations. To avoid coupling between shape functions,  it is important that these shape functions are orthogonal. This means that $\int_\mathbb{X} \varphi_i \varphi_j d \sigma = 0$ for any $i \neq j$ and non-zero otherwise. Given this, the $n$-th order strain expansion can be formulated as,

%% maybe write about what shape functions were used, 

\begin{equation}
\begin{aligned}
    \xi(\sigma,t) \cong & \hspace{5pt}  (B_a \otimes [ \varphi_0 \dots \varphi_N ])q(t)\\ = &  \underbrace{ \begin{pmatrix}
    \varphi_1(\sigma) & \dots  & \varphi_N(\sigma) & \dots     & 0      & \dots  &  0 \\
    \vdots    & \ddots & \vdots    & \ddots    & \vdots & \ddots & \vdots \\
    0         & \dots  & 0         & \dots     & \varphi_0(\sigma) & \dots & \varphi_N (\sigma)
    \end{pmatrix}}_{\Phi(\sigma)} \begin{pmatrix} q_1(t) \\ \vdots \\ q_n(t) \end{pmatrix} +  \begin{pmatrix} \xi_{1,0} \\ \vdots \\ \xi_{n,0}   \end{pmatrix}
    \end{aligned},
\label{eq2:xishape}
\end{equation}

where $\Phi \in \mathbb{R}^{m \times n}$ is a shape function matrix in which $n$ is equal to the amount of modal coordinates, and $m$ the amount of active strains, and $B_a \subseteq \text{span}(\mathbb{I}_6)$ a selection matrix of unconstrained strains. For the planar robot, matrix $B_a$ is equal to,

\begin{equation}
    B_a = \begin{bmatrix}
    0 & 1 & 0 & 0 & 0 & 0 \\
    0 & 0 & 0 & 1 & 0 & 0 \\
    \end{bmatrix}^\top
\end{equation}

as is has one free curvature-torsion strain and one stretch-shear strain, This implies that $m = 2$. The amount of shape functions used to approximate a strain affects the value of $n$. Approximating the two strains with a single shape ($i=0$) gives $n = 2$. Furthermore, the undeformed configuration $\xi_{i,0}$ is given by,

\begin{equation}
    \xi_{i,0} = \begin{bmatrix}  0 & 0 & 0 & 1 & 0 & 0 \end{bmatrix}^\top, 
\end{equation}

and implies that the actuator has an initial length in $\epsilon_1$. direction.

It should be emphasized that when using a single shape function this Cosserat model reduces to the constant curvature model. As can be seen from (\ref{eq2:shapefunction}), all shape functions yield 1 for $i=0$. In this case the modal coordinates $q(t)$ will be equal to $\kappa$ and $\epsilon$, respectively. 

The forward kinematic problem of (\ref{eq2:dgdsigma}) is programmed in \MATLAB \cite{MATLAB2020}. Given modal coordinates $q(t)$ and initial conditions at $\sigma = 0$, the actuator position can be calculated. To reduce computation time, the rotation matrix in (\ref{eq2:g}) is reformulated using quaternion formulation \cite{Boyer2019}. This allows to express any rotation by a vector of length 4, instead of 3 by 3 matrix. Therefore $g(\sigma,t)$ can significantly be reduced from 16 entries, to only 7 entries when using quaternion formulation. Any rotation matrix $R(\sigma,t)$ can be rewritten to quaternions by,


\begin{equation}
\frac{\partial g}{\partial \sigma} \equiv \frac{\partial}{\partial \sigma}    \begin{pmatrix} Q \\ p \end{pmatrix} = \begin{pmatrix} 2 ||Q||^{-1} A(R(Q)K)Q \\ R(Q)p \end{pmatrix},
\label{eq2:Qp}
\end{equation}

where $Q \in \mathbb{R}^4$ is the quaternion representation of rotation matrix $R$, and $R(Q)$ a function mapping a quaternion to rotation matrix representation, $A$ is a function defined as,


\begin{equation}
    A(K) = \begin{pmatrix} 0 & -K_1 & -K_2 & -K_3 \\ K_1 & 0 & -K_3 & \hspace{8pt}K_2 \\ K_2 & \hspace{8pt}K_3 & 0 & -K_1 \\ K_3 & -K_2 & \hspace{8pt}K_1 & 0 \end{pmatrix},
    \label{eq2:AK}
\end{equation}

where $K_i$ correspond to the entries of $\xi(\sigma,t)$ shown in (\ref{eq2:dgdsigma}). In  (\ref{eq2:Qp}) and (\ref{eq2:AK}) the dependency of $\sigma$ and $t$ has been omitted for the sake of readability.

Figure (\ref{fig1:forward_kinematic}) shows the result of the forward kinematic model for a first order approximation. The initial position is obtained for zero curvature and elongation. Here it can be seen that the initial length of the actuator is 64.5 mm. To obtain the deformed modal coordinate $q$ was chosen equal to $[-17,0.1]^\top$. 



\begin{figure}[H]
    \centering
    \includegraphics[width = 0.7\textwidth]{Figures/Chapter2/fkin1701.eps}
    \caption{Initial position and deformed position for a first order shape function approximation of the kinematic model.}
    \label{fig1:forward_kinematic}
\end{figure}

Besides a forward kinematic model, a numerical inverse kinematic solver was programmed. This allows to define a position in planar Cartesian coordinates. The algorithm will minimize the distance between the desired end-effector position and reachable end-effector position given an amount of shape functions. It shows the model's flexibility when using multiple shape functions. The algorithm is further detailed in Appendix \ref{app:chap2}. 

\newpage

%%%%%%%%%%%%%%%%%%%%%%%%%%%%%%%%%%%%%%%%%%%%%%%%%%%%%%%%%%%%%%%%%%%%%%%%
%%%%%%%%%%%%%%%%%%%%%%%%%%%%%%%%%%%%%%%%%%%%%%%%%%%%%%%%%%%%%%%%%%%%%%%%
%%%%%%%%%%%%%%%%%%%%%%%%%%%%%%%%%%%%%%%%%%%%%%%%%%%%%%%%%%%%%%%%%%%%%%%%

\section{Dynamic Modelling}


To study the dynamic behaviour of the soft actuator a dynamic model is created. A relatively simple approximation is used to get insight of the basic dynamics of the actuator. In this we propose a non-linear mass-spring-damper model. The developed model incorporates non-linear mass and non-linear stiffness matrix in an effort to increase model accuracy. The system is approximated by a first order shape function. In this section an analytical expression for the mass matrix is provided. Additionally damping and stiffness matrices are provided. For which the entries are determined in Chapter \ref{chap3}.

Previous section described the kinematics of the soft manipulator. In formulating the forward kinematic problem backbone curve $g(\sigma,t)$ is differentiated with respect to spatial variable $\sigma$. Here, the time dependency of the backbone curve was omitted. In deriving the dynamics of the actuator it is necessary to regard the temporal derivative as,


\begin{equation}
    \frac{\partial g}{\partial t} = g \hat{\eta} \hspace{10pt} \text{with} \hspace{10pt}  \hat{\eta} = \begin{pmatrix} \Omega_\times & V \\ 0_3^\top & 0 \end{pmatrix} \in  \mathfrak{se}(3)
    \label{eq2:dgdt}
\end{equation}

which describes the time-twist field in a local frame at $\sigma$. Here, $\Omega_\times$ is a skew-symmetric matrix expressing angular velocity, and vector $V = (v_1,v_2,v_3)^\top $ contains linear velocities. Therefore, $\eta$ effectively describes velocities among the backbone curve. Again, the skew-symmetric matrix $\Omega_\times$ can be represented as a vector $\Omega = [\omega_1,\omega_2,\omega_3]^\top$ \cite{Sola2018}. These velocities are stored in vector $\eta(\sigma,t) = [\Omega \hspace{5pt} V]^\top$.


A similar expression for $\eta(\sigma,t)$ can be found based on the geometric Jacobian and model coordinate velocities \cite{Boyer2019} \cite{Caasenbrood2020}. This allows to map modal coordinate position to Cartesian coordinates. The Jacobian derivation is based on the principle of mixed partials, and is detailed in \cite{Caasenbrood2020}. Without showing this derivation, the expression is found to be,

\begin{equation}
    \eta(\sigma,t) = \underbrace{\text{Ad}_{g^{-1}} \int_0^{\sigma} \text{Ad}_{g} B_a \Phi(\sigma) d \sigma}_{J(\sigma)} \dot{q}
    \label{eq2:J}
\end{equation}

where $\text{Ad}_g \in \mathbb{R}^{6 \times 6}$ and $\text{Ad}_{g^{-1}}  \in \mathbb{R}^{6 \times 6}$ are the adjoint and inverse adjoint mapping of $g$, respectively \cite{Sola2018}. As mentioned function $g(\sigma,t)$ expresses stress-strain behaviour in a local frame. The adjoint action can be viewed as a mapping from local frame to ground frame ($\sigma$ = 0) or end-effector frame $\sigma = L$. Domain $\sigma$ is bounded by undeformed actuator length $L_0$, as the effect of elongation on the Jacobian is deemed small. Furthermore, $\Dot{q}$ are modal coordinate velocities. 

The geometric Jacobian $J \in \mathbb{R}^{6\times n}$ maps modal coordinate velocity to linear and angular velocities among the backbone curve. This results in a Jacobian matrix which is non-linear with respect to position. 

Based on the Lagrangian method used to describe the soft actuator's  dynamics in \cite{Caasenbrood2020}. An expression for the modal coordinate velocity allows a formulation of the total kinetic energy of the actuator. This kinetic energy is given by, 

\begin{equation}
    \mathcal{T} = \frac{1}{2}\int_0^{\sigma} \eta(\sigma,t)^\top \mathcal{M} \eta(\sigma,t) d \sigma 
    \label{eq2:T}
\end{equation}

where $\mathcal{M} \in \mathbb{R}^{6\times6}$ is a diagonal mass matrix. Up to this point, the system has been viewed as a 1 dimensional backbone curve. To this curve mass properties are assigned. By discretizing this curve we can assign mass properties to this infinitesimal part based on the actuators geometry. To this end, the inertial properties are estimated by viewing each discretized slice as a solid cuboid. Hence, the mass matrix is,
\hl{add a figure of this discretization}

\begin{equation}
    \mathcal{M} = \begin{pmatrix} \frac{1}{12}\rho (w^2 + d^2) & 0 & 0 & 0 & 0 & 0 \\
                                  0 & \frac{1}{12}\rho (w^2 + h^2) & 0 & 0 & 0 & 0 \\
                                  0 & 0 & \frac{1}{12}\rho (d^2 + h^2) & 0 & 0 & 0 \\
                                  0 & 0 & 0 & \rho & 0 & 0 \\
                                  0 & 0 & 0 & 0 & \rho & 0 \\
                                  0 & 0 & 0 & 0 & 0 & \rho \end{pmatrix}\hspace{5pt} \text{with} \hspace{5pt} \rho = \frac{m_{tot}}{L_0}
\end{equation} 




where $m_{tot}$ is the total mass of the actuator. Parameters $w$,$d$ and $h$ the width,depth and height of the discretized cuboid, respectively. The first three diagonal entries relate to curvature of the actuator. The last three entries relate to linear movement.


Substitution of (\ref{eq2:J}) into (\ref{eq2:T}) results in an expression for the non-linear mass matrix as,


\begin{equation}
    M(q) = \int_0^{\sigma} J(\sigma)^\top \mathcal{M}(\sigma)J(\sigma) d \sigma.
\end{equation}


\hl{include the derivation of this based on PDE of the continuous dynamics as in Caasenbrood Thesis} \cite{Caasenbrood2020} 

Now an expression for the mass matrix based on modal coordinates is obtained, we can postulate the model describing the dynamics of the soft actuator. To this, we assume linear damping properties. Furthermore it is known that the material of which the actuator is composed behaves non-linearly. Therefore a non-linear stiffness matrix is introduced. The actuator dynamics that are obtained are then given by,

\begin{equation}
    M(q)\Ddot{q} + D\dot{q} + K(q)q = \tau  hspace{10pt} \text{with} \hspace{10pt} \tau = Hp
    \label{eq2:simp_model}
\end{equation}

where $D \in \mathbb{R}^{2 \times 2}$ and $K(q) \in \mathbb{R}^{2 \times 2}$ are damping and non-linear stiffness matrix, respectively. Vector $\tau \in \mathbb{R}^2$ describes the input moment and force causing curvature and elongation, respectively. The experimental set-up allows for pressure regulation, a to be determined mapping matrix $H \in \mathbb{R}^{2\times2}$ is introduced. This matrix maps input pressure $p \in \mathbb{R}^2$ to force and torque input vector $\tau$. Chapter \ref{chap3} elucidates on the determination of the stiffness parameters and force mapping. Furthermore, system parameters will be presented to calculate the mass matrix. Damping properties will be chosen intuitively and are presented when simulation results are shown.



% hspace{10pt} \text{with} \hspace{10pt} u = Hp

%Since the experimental set-up allows for pressure control, a mapping matrix $H \in \mathbb{R}^{2\times2}$ is necessary. This matrix maps input pressure $p \in \mathbb{R}^2$ to force and torque input vector $u \in \mathbb{R}^2$.


%As mentioned, the work of \cite{Caasenbrood2020} describes the %actuator dynamics based on a Lagrangian formulation. This includes the derivation of the Coriolis and gravitational effects. Here the Coriolis effects are described by,

%\begin{equation}
%    C(q,\dot{q}) = \int_0^\sigma J^\top(\mathcal{M})\Dot{J} + J^\top(\mathcal{M}\text{ad}_\eta - \text{ad}_\eta^\top \mathcal{M})J d \sigma,
%\end{equation}

%where $\Dot{J}$ represents the time derivative of the Jacobian matrix. Furthermore, $\text{ad}_\eta$ represents the adjoint action of velocity vector $\eta$. Additionally gravitational effects are captured by,

%\begin{equation}
%    g(q) = \int_0^\sigma J^\top \mathcal{M} \text{Ad}_{g^{-1}} a_z d\sigma
%\end{equation}

%where $a_z \in \mathbb{R}^6$ is a matrix representing the gravitational acceleration. Therefore $a_z = [0,0,0,-9.81,0,0]^\top$.