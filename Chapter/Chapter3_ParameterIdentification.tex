\label{chap3}

This chapter details the parameter identification of the soft robotic actuator. First, material properties and dimensions of the soft robotic manipulator are provided. Then, determination of actuator stiffness for elongation and curvature explained. Here deformation is related to pressure. Consequently, pressure to force input mapping is detailed. This allows to relate deformation to force, which can be used to determine the non-linear stiffness's. Once these stiffness's are determined, the experimental determination of the pump dynamics is presented. 


Parameter identification is necessary for model simulation as well as experimental validation. Key actuator parameters such as actuator mass and dimensions were presented in an earlier work \cite{berkers}. Since the robotic manipulator is manufactured by an external company, some material properties are classified. To our best knowledge, the Young's Modulus and Poisson ratio are equal to those presented in Table (\ref{tab4:parameters}). These properties allow to calculate shear modulus $G$ as $\frac{E}{2(1+\nu)}$ and bulk modulus $K$ as $\frac{E}{3(1-2\nu)}$. It is assumed that the robotic manipulator deforms non-linearly following Neo-Hookeon behaviour \cite{Caasenbrood2020StiffnessModel}. For consistency with linear elasticity these moduli are reformulated as $C_{10} = \frac{\mu}{2}$ and $D_{1} = \frac{2}{\kappa}$ as used for Finite Element Analysis (FEA) \cite{neohookean}. The geometrical properties are presented in Figure \ref{fig3:dim}.


\begin{table}[H]
    \centering
    \caption{Actuator properties}
    \begin{tabular}{|c|c|c|} \hline
      \textbf{Parameter}   &  \textbf{Value} & \textbf{Unit} \\ \hline
      Mass $m$             &    0.0332       & $[kg]$ \\ 
      Nominal length $L_0$ &    0.0644       & $[m]$  \\ 
      Actuator width  $w$     &    0.0664       & $[m]$  \\
      Lever length $r$     &    0.01256      & $[m]$  \\ 
      Young's Modulus $E$  &    69           & $[MPa]$\\ 
      Poisson ratio $\nu$ &    0.45         & $[-]$ \\ \hline
    \end{tabular}
    \label{tab4:parameters}
\end{table}

\begin{figure}[H] 
    \begin{minipage}[b]{0.49\linewidth}
     \centering
    \includegraphics[width=\linewidth]{Figures/Chapter3/dimensions.png} 
    \caption{Schematic overview of the undeformed actuator with its dimensions.} 
    \label{fig3:dim} 
       \end{minipage} 
    \begin{minipage}[b]{0.49\linewidth}
     \centering
    \includegraphics[width=0.7\linewidth]{Figures/Chapter3/undeformed2.png} 
    \caption{Undeformed meshed Finite Element model of the planar actuator.} 
    \label{fig3:FemModel} 
    \end{minipage} 
\end{figure}

%%%%%%%%%%%%%%%%%%%%%%%%%%%%%%%%%%%%%%%%%%%%%%%%%%%%%%%%%%%%%%%%%%%%%%%%
%%%%%%%%%%%%%%%%%%%%%%%%%%%%%%%%%%%%%%%%%%%%%%%%%%%%%%%%%%%%%%%%%%%%%%%%
%%%%%%%%%%%%%%%%%%%%%%%%%%%%%%%%%%%%%%%%%%%%%%%%%%%%%%%%%%%%%%%%%%%%%%%%

\section{Finite Element Analysis (FEA)}


In Chapter \ref{chap2} we derived that two non-linear stiffness's need to be determined. Namely, non-linear curvature stiffness $K_\kappa(q)$ and the elongation stiffness $K_\epsilon(q)$. To determine these stiffness properties, finite element software \verb+Abaqus/CAE+ is used. This software allows to study deformation of the actuator under various loads. Stiffness can be approximated by relating applied forces to the magnitude of deformation. Essentially, we want to retrieve the modal coordinates $q$ of the deformed actuator for a range of applied pressures. These pressures can be related to some effective moment/force via some mapping $H$, as detailed in Chapter \ref{chap2}. With the obtained modal coordinates, and this moment/force we can determine the soft robot's stiffness's. During the analysis we assume no coupling between curvature and elongation, therefore each stiffness can be determined separately. To this end, two different analysis are performed. The first analysis focuses on curvature, the second analysis targets elongation. Before these analysis's are discussed the used FEM model and applied constraints are explained. Consider Figure \ref{fig3:FemModel} which shows the meshed model used in the finite element software. A mesh refinement analysis is done to determine an accurate mesh size, see Appendix \ref{app:chap3}. For both analysis's, the bottom plate of the actuator is constrained in all directions of motion and rotation. Furthermore, the out-of-plane motion of the actuator is not considered since the body is symmetric, and applied loads act perpendicular to this motion. Also, gravitational effects are omitted. First we consider the curvature analysis and the extraction of the modal coordinates after deformation. Then the elongation analysis is explained, where the same methodology is used. 


\subsection{Curvature analysis}

The curvature analysis allows to determine rotational stiffness of the robot manipulator. During this analysis a single bellow is pressurized, the induced force causes a moment around the center of the soft robot. This will case the soft robot to curve and elongate. Figure \ref{fig3:schematiccurvature} shows the deformation the soft robot undergoes during a curvature simulation. Here the undeformed actuator is visualized in blue. For the black actuator the left bellow is pressurized to 60 kPa.



\begin{figure}[H]
    \centering
\begin{minipage}{0.5\textwidth}
        \centering
        \includegraphics[width=0.695\textwidth]{Figures/Chapter3/curvature.png} 
        \caption{Curvature analysis, undeformed soft robot in blue and deformed robot in black for curvature analysis. }
        \label{fig3:schematiccurvature}
    \end{minipage}\hfill
    \begin{minipage}{0.5\textwidth}
        \centering
        \includegraphics[width=\textwidth]{Figures/Chapter3/rotation60kpa.eps} 
        \caption{Post-processed image of the curvature analysis. The nodes that form the backbone curve clearly isolated.}
        \label{fig3:nodalcurvatrue}
    \end{minipage}
\begin{minipage}{0.5\textwidth}
        \centering
        \includegraphics[width=\textwidth]{Figures/Chapter3/rot60kpa.eps}
        \caption{Inverse kinematic fit for an curvature analysis. Left bellow pressurized to 60kPa.}
        \label{fig3:nodalfitcurv}
    \end{minipage}\hfill
    \begin{minipage}{0.5\textwidth}
        \centering
        \includegraphics[width=\textwidth]{Figures/Chapter3/curvanalysiscurveps.eps} 
        \caption{Curvature analysis, elongation and curvature as function of pressure.}
        \label{fig3:rotationvspressure}
    \end{minipage}
\end{figure}



After applying loads to the finite element model, nodal displacements can be outputted from $\verb+Abaqus+$. The acquired data is post-processed in \MATLAB to estimate the elongation and curvature. A post-processed image of the deformation is shown in Figure \ref{fig3:nodalcurvatrue}. This image is created by isolating specific nodes, necessary to reconstruct the backbone curve. In this figure, top and bottom plate of the actuator are clearly visible. Using \MATLAB function \verb+affine_fit.m+ \cite{affinefit} the orientation of both planes with their normal vector can be obtained. Furthermore, this function determines the weighted average of all nodes of the top plate. Which coincides with the end-effector position of the actuator, $\Bar{x}_end$. Lastly, the nodes that are situated at the exact geometrical mid over the longitudinal axis are isolated. These clusters of nodes are used to reconstruct the backbone curve. 


The post-processed image can be used for kinematic model fitting. This is the last step to obtain modal coordinates $\epsilon$ and $\kappa$ of the deformed soft robot. Figure \ref{fig3:nodalfitcurv} shows this kinematic model fit. Here the black cross again indicates the end-effector position. Initially, the kinematic fitting is done with the aid of the developed inverse kinematic model as detailed in Appendix \ref{app:chap2}. An inverse kinematic solution can be found based on the end-effector position and rotation. The solution to this kinematic fitting procedure is indicated by the purple curve. This result was deemed insufficiently accurate. Therefore an optimization method is proposed. This optimization method takes into account the curvature of the entire backbone curve. For each cluster of nodes among the backbone the weighted average is calculated, indicated by a blue cross in the figure. The $x$ and $y$ position of these nodes are stored in matrix $\Bar{x}_{mid} \in \mathbb{R}^{2\times 10}$, where $10$ corresponds to the amount of node clusters. Furthermore, the length of the backbone curve is determined, which corresponds to the elongation of the soft robot. This is done using \MATLAB function \verb+arc_length.m+ \cite{arclength}. This functions uses matrix $\Bar{x}_{mid}$ to approximate the arc length using splines. This estimated arc length is represented as $\Bar{l}_{est} \in \mathbb{R}^+$. The modal coordinates can then be found through optimization with \verb+fmincon.m+. The minimization problem that follows is,


\begin{equation}
\begin{aligned}
\min_{q} \hspace{5pt}  q^\top Q q  + \sum_{i=1}^{10}\Big(\psi_1 ||\Bar{x}_{mid,(:,i)} - f(q)_{mid,(:,i)})|| \Big) +   \psi_2(\Bar{x}_{end}  - f(q)_{end})^2 +  \psi_3(\Bar{l}_{est} - & l_{f(q)})^2  \\ 
\text{s.t.} \hspace{5pt} \epsilon - 1 > 0
\end{aligned}
\label{eq3:optim}
\end{equation}


where $f(q)$ is a function describing the forward kinematics based on modal coordinates $q$. Subscripts $mid$ and $end$ correspond the calculated middle nodes and end-effector position, respectively. Scalar $l_f(q) \in \mathbb{R}^+$ is the estimated actuator length based on the output of function $f(q)$. The summation is over 10 mid nodes which corresponds to the length of matrix $\Bar{x}_mid$. Furthermore, $Q$ is $\text{diag}([0.001,0.001])$ is a weighing matrix used to penalize the value of the modal coordinates. Additional weighing factors $\psi_i \in \mathbb{R}^+$ and $i \in \{1,2,3\}$ are applied to penalize individual terms and equal to $5e4$,$5e4$ and $1e4$, respectively. This weighing factors can used to scale individual error terms, and stress relative importance. The only imposed constrained affects the elongation of the manipulator. Physically this constraint implies that the actuator can shorten in length, but the length can not become negative. As a note, for a single shape-function approximation, solving the minimization problem only results in little better modal coordinate estimates. For studying more complex actuator deformations, that involve estimations with multiple shape functions, this optimization is necessary. An inverse jacobian search singularly based on end-effector position will not be enough then. Some reconstruction of the backbone curve will be necessary for such cases. The modal coordinates belonging to the deformation of Figure \ref{fig3:nodalfitcurv} find through optimization are $q = [\kappa,\epsilon]^\top = [-14,0.24]^\top$, which corresponds to an elongation of 24\% and curvature of 14$\frac{1}{m}$ in clockwise direction.


Above described inverse kinematic optimization is ran for multiple finite element analysis conducted for 10 other bellow pressures. The results are displayed in Figure \ref{fig3:rotationvspressure}. Here the obtained curvature and elongation are plotted as a function of bellow pressure. It can be seen that the curvature and elongation rates increase by increments in pressure. Also the non-linear relation between pressure and deformation is visible, as we do not observe a linear relationship between pressure and curvature. 




\subsection{Elongation analysis}
\label{subsecelong}

The elongation analysis aims to measure pure elongation of the soft actuator. This allows for determining elongation stiffness, as rotating effects of the top of the robot are small. ``Pure'' elongation is induced by pressurizing both bellows equally. Figure \ref{fig3:schematicelong} shows the deformation the actuator undergoes during an elongation experiment. Here the undeformed actuator is visualized in blue. For the black actuator both bellows are pressurized to 60 kPa. 

As for the curvature analysis, a post-processed image of the deformation is shown in Figure \ref{fig3:nodalelong}. The optimization of (\ref{eq3:optim}) is carried out, to determine modal coordinates. The results of this optimization is shown in Figure \ref{fig3:nodalfitelong}. The modal coordinates belonging to this fit are  $q = [\kappa,\epsilon] = [0.06, 0.45]^\top$, showing a negligible curvature and elongation of 45\%. Furthermore, this figure shows that for elongation analysis, the developed inverse kinematic algorithm (Appendix \ref{app:chap2}) already yields decent results.  

Repeating this experiment for a set of 10 pressures, a relation between elongation and pressure can be found. Figure (\ref{fig3:elongationvspressure}) shows the relation between elongation function of pressure. Additionally, the curvature of the manipulator is shown for various pressures. It can be seen that pressurizing the bellows equally results in a near pure elongation. The maximum obtained curvature is equal to 0.07 $\frac{1}{m}$, which is equivalent to a rotation of the upper flange of 0.32 degrees. This allows to determine the elongation stiffness as effects of rotation are deemed negligible. The elongation rate decreases as pressure increases, this is the result of a non-linear material parameters. As a remark, the found elongation for the elongation analysis is about double to those found for curvature analysis. For instance, the found elongation for the elongation experiment at 90 kPa is $0.55$, whereas for the curvature analysis the elongation at this pressure is $0.27$. This observation tells that the curvature and elongation are largely decoupled.



\begin{figure}[H]
    \centering
\begin{minipage}{0.5\textwidth}
        \centering
        \includegraphics[width=0.53\textwidth]{Figures/Chapter3/elongation.png} 
        \caption{Elongation analysis, undeformed soft robot in blue and deformed robot in black. }
        \label{fig3:schematicelong}
    \end{minipage}\hfill
    \begin{minipage}{0.5\textwidth}
        \centering
        \includegraphics[width=\textwidth]{Figures/Chapter3/elongation60good.eps} 
        \caption{Post-processed image of an elongation analysis. The nodes that form the backbone curve clearly isolated.}
        \label{fig3:nodalelong}
    \end{minipage}
\begin{minipage}{0.5\textwidth}
        \centering
        \includegraphics[width=\textwidth]{Figures/Chapter3/elong60kpa.eps}
        \caption{Inverse kinematic fit for an elongation analysis. Both bellows pressurized to 60kPa.}
        \label{fig3:nodalfitelong}
    \end{minipage}\hfill
    \begin{minipage}{0.5\textwidth}
        \centering
        \includegraphics[width=\textwidth]{Figures/Chapter3/elonganalysiscurveps.eps} 
        \caption{Elongation analysis, elongation and curvature as function of pressure. As can be seen curvature is negligibly small.}
        \label{fig3:elongationvspressure}
    \end{minipage}
\end{figure}


At this point we have derived modal coordinates as function pressure. However, to determine the stiffness this pressure needs to be mapped to force resulting in elongation and a moment causing curvature. This is discussed in Section \ref{sec3:InputMapping}.



%%%%%%%%%%%%%%%%%%%%%%%%%%%%%%%%%%%%%%%%%%%%%%%%%%%%%%%%%%%%%%%%%%%%%%%%
%%%%%%%%%%%%%%%%%%%%%%%%%%%%%%%%%%%%%%%%%%%%%%%%%%%%%%%%%%%%%%%%%%%%%%%%
%%%%%%%%%%%%%%%%%%%%%%%%%%%%%%%%%%%%%%%%%%%%%%%%%%%%%%%%%%%%%%%%%%%%%%%%



\section{Input Mapping}
\label{sec3:InputMapping}

Stiffness is the ratio between applied force and elongation. To determine stiffness for elongation and curvature, applied forces and direction of deformation is necessary. For the physical set-up, individual bellow pressure $p_i \in \mathbb{R}^{\geq 0}$ with $i$ $\mathbb{N} \in \{1,2\}$ can be regulated. Therefore force input vector $\nu(p,t)$ is mapped to pressure by some mapping matrix $H \in \mathbb{R}^{2 \times 2}$, as discussed in Chapter \ref{chap2}.  The relation between force input and pressure is given as, 

\begin{equation}
   \nu(p,t) =   \begin{bmatrix} M \\ F \end{bmatrix}     = \underbrace{\begin{bmatrix}  H_{1,1} & H_{1,2} \\ H_{2,1} & H_{2,2} \end{bmatrix}}_{H}         \begin{bmatrix}  p_1 \\ p_2 \end{bmatrix}, \label{eq3:H}
\end{equation}

where entries of $H$ are to be determined. Our aim is to decouple rotation and elongation. Therefore, it is important to understand that $M$ causes curvature of the soft robot. Whereas, $F$ results in elongation. Therefore, the entries of $H_{1,1}$ and $H_{1,2}$ represent an effective surface area multiplied by level on which the force acts. Entries $H_{2,1}$ and $H_{2,2}$ represent only represent an effective surface area. Due to symmetry properties it must hold that $H_{2,1} = H_{2,2}$ and $H_{1,1} = -H_{1,2}$.

First, the relation between pressure and force tried to be obtained. This will be done by using finite element software as mentioned. The idea is to map pressure to force using elongation. Pressurizing both bellows equally, resulted in almost ``pure" elongation. Effectively, the same deformation can be achieved by applying an equally distributed force on the top plate of the actuator. To this end, simulations for a set of forces applied to the top of actuator are simulated. This FEA follows the same procedure as for the elongation analysis as discussed in Section \ref{subsecelong}. The results for this simulation are presented in Figure (\ref{fig3:forcemap}). The horizontal axis show the modal coordinates, determined using the inverse kinematic optimization as discussed previously. The markers in blue correspond to the modal coordinates found for the elongation analysis. The red markers indicate modal coordinates as found by the force analysis. It is clear that pressurizing both bellows or a distributed force on the upper flange result in comparable deformation. The scaling of the of the vertical axis's already reveal that a linear relation can be found.


\begin{figure}[H]
    \centering
\begin{minipage}{0.5\textwidth}
        \centering
        \includegraphics[width=\textwidth]{Figures/Chapter3/forcepressuremodal.eps}
        \caption{Modal coordinates $\kappa$ and $\epsilon$ for elongation analysis and force analysis.}
        \label{fig3:forcemap}
    \end{minipage}\hfill
    \begin{minipage}{0.5\textwidth}
        \centering
        \includegraphics[width=\textwidth]{Figures/Chapter3/pressureforceelongation.eps} 
        \caption{Elongation expressed for force analysis and force with input mapping.}
        \label{fig3:forcetopressure}
    \end{minipage}
\end{figure}

The linear mapping coefficient, which maps the pressure to a force, can be found by solving a least squares problem as,

\begin{equation}
\min_{H_{2,2}} \sum_{i=1}^{samples} (F_i - 2 H_{2,2} p_i)^2
\label{eq3:forcefitting}
\end{equation}

where $i$ indicates each sample point of the total 11 sample points. The effective pressure area is found to be equal to 0.1462 $m^2$. Since $H_{2,1} = H_{2,2}$ the mapping of pressure in kPa to force in Newton is obtained. Figure (\ref{fig3:forcetopressure}) shows the results of the input mapping for the elongation. It can be seen that the pressure maps fairly accurate to the force.

Now the effective pressure area has been determined, the mapping to the moment can be calculated. It is assumed that the force acts on the top plate at a given radius form the backbone curve. The lever on which the force acts is equal to 12.56 $mm$, as can be seen in Figure \ref{fig3:dim}. This allows to calculate $H_{1,1}$ and $H_{2,1}$ by multiplying $H_{2,1}$ with this distance. The mapping matrix $H$ then follows as,

\begin{equation}
    H =  \begin{bmatrix} 1.8245e-3 & -1.8245e-3 \\
    0.14526 & 0.14526\end{bmatrix}.  
\end{equation}

At this point, the modal coordinates for the deformed soft robot for curvature and elongation analysis have been determined. Furthermore, we have determined a mapping to relate pressure to force and moment. This allows us to derive the curvature and elongation stiffness of the soft robot. 

%%%%%%%%%%%%%%%%%%%%%%%%%%%%%%%%%%%%%%%%%%%%%%%%%%%%%%%%%%%%%%%%%%%%%%%%
%%%%%%%%%%%%%%%%%%%%%%%%%%%%%%%%%%%%%%%%%%%%%%%%%%%%%%%%%%%%%%%%%%%%%%%%
%%%%%%%%%%%%%%%%%%%%%%%%%%%%%%%%%%%%%%%%%%%%%%%%%%%%%%%%%%%%%%%%%%%%%%%%



\section{Stiffness properties}

The force mapping and the modal coordinates obtained through inverse kinematic optimization can be used to determine elongation and curvature stiffness. A non-linear trend in elongation and curvature as function of pressure was already observed. Therefore, it is assumed that the stiffness can be captured by the hyper-elastic stiffness model as formulated in \cite{Caasenbrood2020StiffnessModel}. This non-linear stiffness model poses that stiffness $K(q_i) \in \mathbb{R}^{\geq 0}$ and meets condition $K_{min} < K(q_i) < K_{max}$. This means that the actuator stiffness is asymptotic to $K_{max}$ and $K_{min}$.  The elongation stiffness is given by,

\begin{equation}
    K_\epsilon(\alpha,\epsilon) =  \alpha_1 + \alpha_2 [\tanh({\alpha_3 \epsilon})^2 -1],
\end{equation}


where $\alpha \in \mathbb{R}^3$ are positive stiffness parameters to be found. It is assumed that negative pressures, e.g. creating a vacuum, result in equal elongation yet in opposite direction. Hence, the amount of sample points is doubled. The stiffness parameters can be found by solving the non-linear constraint optimisation described by,


\begin{equation}
\begin{aligned}
\min_{\alpha_1,\alpha_2,\alpha_3} \hspace{5pt} \sum_{i=1}^{samples}\Big(F_i -  (\alpha_1 + \alpha_2 [\tanh({\alpha_3 \epsilon_i})^2 -1])\epsilon_i\Big)^2    \\ 
\text{s.t.} \hspace{5pt} \alpha_1 > \alpha_2 > 0 \\
\alpha_3 > 0 \\ ,
\label{eq3:Keopt}
\end{aligned}
\end{equation}

which objective is to minimize the the sum of the errors between the mapped force, and force resulting from the stiffness model based on the elongation. The results of this optimization are shown in Table \ref{tab3:stiffnessparameters}.

In determining the curvature stiffness the procedure is repeated. Consider the curvature stiffness given by,

\begin{equation}
    K_\kappa(\beta,\kappa) =  \beta_1 + \beta_2 [\tanh({\beta_3 \kappa})^2 -1],
\end{equation}

where $\beta \in \mathbb{R}^3$ contains positive stiffness parameters. Likewise, these parameters can be found by solving,

\begin{equation}
\begin{aligned}
\min_{\beta_1,\beta_2,\beta_3} \hspace{5pt} \sum_{i=1}^{samples}\big(M_i -  (\beta_1 + \beta_2 [\tanh({\beta_3 {\kappa_i}})^2 -1]){\kappa_i}\Big)^2    \\ 
\text{s.t.} \hspace{5pt} \beta_1 > \beta_2 > 0 \\
\beta_3 > 0 \\ 
\label{eq3:Kkopt}
\end{aligned}
\end{equation}

where the objective is to minimize the sum of errors between the mapped moment and the curvature stiffness as given by the model. The resulting stiffness of this optimization are shown in Table (\ref{tab3:stiffnessparameters}) as well. 

\begin{table}[H]
    \centering
        \caption{Parameters for hyper-stiffness model.}
\begin{tabular}{|c|c|c|c|} \hline
            &  $i = $ 1      &    $i = $    2   &  $i = $ 3  \\ \hline
   $\alpha_i \hspace{2pt}[N]$    &    1.3936e+3    & 1.3776e+3    & 2.7865e-1 \\ \hline
   $\beta_i \hspace{2pt}  [Nm^2] $     &  3.0322 & 3.0309    &  3.3755e-3\\ \hline
\end{tabular}
    \label{tab3:stiffnessparameters}
\end{table}

The stiffness model can be plotted as a function of applied force and moment. These results are shown in Figure \ref{fig3:elongvsforce} and \ref{fig3:curvsmoment} for force and moment, respectively. The model shows that force and moment applied for the FEA analysis well correspond to the obtained stiffness function. 


\begin{figure}[H]
    \centering
\begin{minipage}{0.5\textwidth}
        \centering
        \includegraphics[width=\textwidth]{Figures/Chapter3/mappedforcevselongation.eps}
        \caption{Fitted stiffness model for elongation.}
        \label{fig3:elongvsforce}
    \end{minipage}\hfill
    \begin{minipage}{0.5\textwidth}
        \centering
        \includegraphics[width=\textwidth]{Figures/Chapter3/mappedmomentvscurvature.eps} 
        \caption{Fitted stiffness model for curvature.}
        \label{fig3:curvsmoment}
    \end{minipage}
\end{figure}


At this point it should be emphasized that during experimental analysis the soft actuator, on which this parameter study is based, was too porous. This allowed too much air to escape and therefore was not suitable for further use. Therefore, the methodology of acquiring stiffness data and pressure mapping should be valued instead of found parameters. During experimental analysis a geometric equivalent planar actuator is used. 



\section{Pump Dynamics}





Besides soft robot actuator properties also the actuator properties are involved. To this end the pump dynamics are determined. In the proposed control architecture, the Jacobian controller is accompanied by a pressure controller. To this end, the pump dynamics need to be determined. Initially, we determine the pump dynamics by  connecting the pump directly to the pressure sensor. This yielded poor performance once connected to the soft actuator. To suppress noise levels and capture the actual system dynamics better the set-up is slightly adapted. Furthermore we make the following assumption:


\begin{theorem}
Both air pumps have the equal pump characteristics, therefore the system dynamics of a single air pump need to be determined.
\end{theorem}


To determine the pump dynamics the pump is connected directly to the pressure sensor using a short flexible silicone hose. The pump is powered by an electric 12V DC motor, and uses membranes to create pressure. The pump characteristics are obtained by observing the systems response for various volt step inputs between 1 and 6 Volt. Although the pumps are able to operate at 12 Volt, only a maximum step input of 6V is possible for this set-up. The pressure sensor measures absolute pressures between 0 and 25 PSI, which is equivalent to a maximum of 172 kPa. Depending on weather conditions, the maximum theoretically measurable pressure is around 72 kPa. However, experiments have shown results up to 80 kPa are possible. The resulting step responses are shown in Figure (\ref{fig1:pump_dynamcis}).

\begin{figure}[H]
    \centering
    \includegraphics[width = 0.6\textwidth]{Figures/Chapter3/stepresponsdirect16V.eps}
    \caption{Step response for volt step input between 1V and 6V.}
    \label{fig1:pump_dynamcis}
\end{figure}

Multiple observations can be made based on the step response of Figure (\ref{fig1:pump_dynamcis}). A first glance reveals first order system behaviour as  pressure increase seems to be proportional to the actual pressure. However, the steady-state pressure is not proportional to the input volt for all step responses. The reason for this behaviour is thought to be friction. For 1 and 2 volt input static friction dominates, resulting in no to very small pressure increase. At 3 volt the transition between dry and kinetic friction is observed. The first order system characteristic become better visible, although the pressure response is chattering. Furthermore, we see a rise time of comparable order for step inputs of 4V and above. After around 0.5 seconds the steady-state pressure is reached. For all step inputs oscillations are observed around the steady-state pressure. This phenomena is caused by the fact that there is almost no passive leakage from the small control volume. When the pump valves open and air is pressed into the system the air waves are created. For the step input of 5 volt the valve dynamics can be clearly seen, as the dense green spikes.

An improved pump model can be obtained by altering the set-up. To this end, the actuator and an air vessel are connected to the system. Previous research showed that adding an air vessel will reduce oscillatory behaviour, at the cost of bandwidth \cite{proper}. The air vessel will increase the control volume of the entire system. When the valves open and additional air is pressed in the system, the relative pressure change is smaller. Additionally, when the actuator expands the relative volume increase will also be smaller. This will cause the system to respond slower to a step input. Furthermore, the maximum achievable pressure is lower. The addition of air vessel and actuator will increase the passive leakage of the system.

For the adapted system the step response is determined for volt steps between 2 and 12 volt. The results are shown in Figure \ref{fig3:pump_dynamics_adapted}.

\begin{figure}[H]
    \centering
    \includegraphics[width = 0.6\textwidth]{Figures/Chapter3/step212V.eps}
    \caption{Step response for volt step input between 2V and 12V for the system including air vessel and actuator.}
    \label{fig3:pump_dynamics_adapted}
\end{figure}


Based on the step responses of Figure (\ref{fig3:pump_dynamics_adapted}) we try to capture the pump dynamics in a model. The time response of a first order linear system responding to a step input is given by, 

\begin{equation}
    p(t,V) = K(1-e^{-t/\tau})V,
    \label{eq3:firstordermodel}
\end{equation}

where $p$ is the pressure at time $t$ in kPa. DC-gain $K$ [kPa/V] determines the maximum pressure, time constant $\tau_s$ [1/s] determines the growth rate of the exponential function. Furthermore, $V$ is the input volt. 

An attempt to fit the linear first order model of (\ref{eq3:firstordermodel}) did not result in an accurate description of the pump dynamics for all step inputs. Therefore an alternative description of the pump dynamics is proposed. In this description DC-gain $K$ is replaced by a nonlinear function $K(V)$. Additionally, a linear function $\tau_s(V)$ is proposed. This adaption makes the original model non-linear. Additionally, we are aware that by introducing these functions the terminology time `constant' is used incorrectly.


Based on (\ref{eq3:firstordermodel}) an indication of the gradient between constant $K$ and $V$ can be estimated. The DC-gain can be found by dividing the steady-state pressure by its corresponding volt input. The steady-state pressure for each step input is estimated by taken the mean over the last 10 seconds of data. A similar approach is done for determining time constant $\tau_s$. For a first order system it can be proven that this time constant is equal to time at which 63\% of the steady state value is reached. The corresponding relations for $K(V)$ and $\tau_s(V)$ are shown in Figure \ref{fig3:Kest} and Figure \ref{fig3:tauest}, respectively.
\newpage

\begin{figure}[H]
\begin{minipage}[b]{0.48\linewidth}
\centering
\includegraphics[width=\textwidth]{Figures/Chapter3/Kest.eps}
\caption{Experimental estimate of `constant' $K$ and fourth order fit.}
\label{fig3:Kest}
\end{minipage}
\begin{minipage}[b]{0.48\linewidth}
\centering
\includegraphics[width=\textwidth]{Figures/Chapter3/tauest.eps}
\caption{Experimental estimate of time `constant' $\tau$ and linear fit.}
\label{fig3:tauest}
\end{minipage}
\end{figure}

Figure \ref{fig3:Kest} shows a fourth-order polynomial fit through DC-gain $K$. For step inputs larger than 8 volt the DC-gain drops. This indicates that the air pump is less efficient for these volt inputs. This could be caused an increased temperature of the air pump. Another explanation for this behaviour is an increased passive leakage of the system, as steady-state pressure is higher for high volt inputs. Up till 5V the pumps do not adhere to what is expected. The DC-gains found in this region indicate inefficient pump behaviour. It is believed that this is caused by friction.

Figure \ref{fig3:tauest} shows the estimated time constants, with a linear fit. Clearly the estimated time constant for 2 and 3 volt do not follow this linear trend line. Therefore, this data point where excluded from the linear fit. A possible explanation for these outliers is the friction acting at these voltage inputs, and are therefore not reliable. Furthermore, we see the time constant decreasing for a increasing volt. This means that the pump has a faster response for higher volt input. This can make sense as the total power supplied to the air pumps is higher. 


The obtained measurement fits for $K$ and $\tau_s$ can be substituted in (\ref{eq3:firstordermodel}) resulting in an expression as,

\begin{equation}
    p(t,V) = K(V)(1-e^{-t/\tau_s(V)})V.
    \label{eq3:firstodernonlinearmodel}
    \end{equation}

The derived first order model of (\ref{eq3:firstodernonlinearmodel}) can be plotted against the experimental results of Figure (\ref{fig3:pump_dynamics_adapted}), this is shown in Figure (\ref{fig3:expvsfitpres}).

\newpage

\begin{figure}[H]
    \centering
    \includegraphics[width = 0.7\textwidth]{Figures/Chapter3/expfit.eps}
    \caption{Experimental results and the derived nonlinear pressure model.}
    \label{fig3:expvsfitpres}
\end{figure}

Above figure shows that the derived model captures the behaviour of the system with actuator and air vessel accurately. The steady-state pressure is close to the experimental value for all volt step inputs. As expected, the rise time for the 1 and 2 volt step inputs are not captured that well by this model.

The obtained dynamic response is arguable from a physical point of view. The dynamic model for the air pumps is fit to a first order linear system by introducing non-linearity's. Therefore the obtained model is very specific and does not allow for a change in parameters. Therefore this model can not be used in a changed set-up. Furthermore it should be noted that the fourth order fit used to describe the DC-gain is found negative for Volt input equal to 0. From a physical perspective this is not possible. At 0 volt input, the DC-gain should be equal to 0. Therefore the model is only deemed valid in the volt input range where $K$ is positive.

\hl{In this section sinusoidal response of the pump model should be added for experimental verification}


