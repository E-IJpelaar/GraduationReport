
To develop a suitable model based controller, the developed non-linear model should first be validate. This is done using a planar 2 degree-of-freedom soft robot. This robot has a translational and rotational degree of freedom. Rotation is only dependent on the difference in bellow length, not absolute bellow length. To measure rotation, an Inertial Measurement Unit (IMU) is connected at the tip of the robot. The position of the bellows can be tracked with an optical tracking system. From this data, elongation and curvature of each individual bellow can extracted. This allows validation of the dynamic model.

\section{Model validation for the planar soft actuator}

\subsection{Forward kinematics}

The forward kinematics of the robot manipulator are derived by using the model formulated in \cite{Caasenbrood2020}. Here, the soft robotic manipulator is regarded as a continuous deformable structure with infinite degrees of freedom. A one-dimensional Cosserat beam model is used to describe the configuration space of the robot. This one-dimensional beam can be regarded as curve through the centre of the robot in longitudinal direction. This curve describes the configuration of the soft robot as a function of space and time. 

Consider a spatial coordinate $\sigma \in \mathbb{X}$ within the bounded domain $\mathbb{X} \in [0,l] \subset \mathbb{R}$, and a temporal coordinate $t \in  \mathbb{T}$ with $\mathbb{T} \subseteq \mathbb{R}$. This allows to describe the position $p(\sigma,t) \in \mathbb{R}^3$ and rotation $R(\sigma,t) \in \mathbb{SO}(3)$ 
for any point $\sigma$ and instance $t$ of the centre of the soft manipulator by,


\begin{equation}
    g(\sigma,t) := \begin{pmatrix}  R(\sigma,t) & p(\sigma,t) \\ 0_3^\top & 1 \end{pmatrix} \in \mathbb{SE}(3),
    \label{eq2:g}
\end{equation}

where $\mathbb{SE}(3)$ is the Lie group of rigid body transformations in $\mathbb{R}^3$ \cite{Sola2018}. The partial derivatives of $g$ can be described by two vector fields in the Lie algebra $\mathfrak{se}(3)$. Here $\partial (.)/\partial(\sigma)$ and $\partial (.)/\partial(t)$ will be denoted by a `prime' and `dot' for position and time, respectively. The partial derivative with respect to time of (\ref{eq2:g}) results in the time-twist field as described by,

\begin{equation}
    \frac{\partial g}{\partial t} = \dot{g} = g\hat{\eta} \Longrightarrow \hat{\eta} := g^{-1}\dot{g} = \begin{pmatrix} \Omega_\times & V \\ 0_3^\top & 0 \end{pmatrix} \in  \mathfrak{se}(3),
    \label{eq2:dotg}
\end{equation}

where $\Omega = (\omega_1,\omega_2,\omega_3)^\top$ and $V = (v_1,v_2,v_3)^\top$ express vectors of angular velocity and linear velocity, respectively. The skew-symmetric matrix $\Omega_\times$ can be related to $\mathfrak{so}(2) \cong \mathbb{R}^3$ given the mapping $\Omega \mapsto \Omega_\times$ \cite{Sola2018}. Here, vector field $\eta(\sigma,t)$ describes an infinitesimal local transformation by a frame at position $\sigma$ between infinitesimally small $t$ and $t+\Delta t$. Following the same logic, the partial derivative of (\ref{eq2:g}) with respect to space results in the space-twist field defined as,


\begin{equation}
   \frac{\partial g}{\partial \sigma} =  g' = g\hat{\xi} \Longrightarrow \hat{\xi} := g^{-1}g' = \begin{pmatrix} K_\times & E \\ 0_3^\top & 0 \end{pmatrix} \in  \mathfrak{se}(3),
    \label{eq2:gaccent}
\end{equation}

in which $K = (k_1,k_2,k_3)^\top$ and $E = (\epsilon_1,\epsilon_2,\epsilon_3)^\top$ represent vectors containing the curvature-torsion strain and stretch-shear strain, respectively. Likewise, vector field $\xi(\sigma,t)$ describes an infinitesimal local transformation by a frame at time instance $t$ between infinitesimally small $\sigma$ and $\sigma+\Delta \sigma$. Because $\mathfrak{se}(3) \cong \mathbb{R}^6$ and with mapping $\eta \mapsto \hat{\eta}$, (\ref{eq2:dotg}) and (\ref{eq2:gaccent}) can be formulated as a column vector in $\mathbb{R}^6$ as,


\begin{equation}
    \eta(\sigma,t) = \begin{pmatrix} \Omega \\ V \end{pmatrix}; \hspace{20pt} \xi(\sigma,t) = \begin{pmatrix} K \\ E \end{pmatrix}.
    \label{eq2:etaxi}
\end{equation}




For solving the forward kinematics only position of the soft robotic manipulator is of interest. Hence, first order partial differential equation (PDE)

\begin{equation}
    \frac{\partial g}{\partial \sigma} = g \hat{\xi},
    \label{eq2:dgdsigma}
\end{equation}

needs to be solved. To solve this PDE, it is transformed to an ordinary differential equation (ODE) by exploiting the Galerkin reduction method. Here, the infinite dimensional system is projected onto a subspace of finite dimension that contains basis elements of the expected solution. By reducing the dimensionality of the system, higher order dynamics are not captured in the model and thus robustness should be taken into account. To transform PDE (\ref{eq2:dgdsigma}), the components of the strain field $\xi := (g^{-1}g')$ are approximated using a finite amount of shape functions as,

\begin{equation}
    \xi_i(\sigma,t) \cong \sum_{i=1}^N \varphi_i(\sigma)q_i(t) + \xi_{i,0}(\sigma), \hspace{20pt} \forall \sigma \in \mathbb{X}, t \in \mathbb{T}
\end{equation}

in which, $\xi_{i,0}$ corresponds to the undeformed configuration of the robotic manipulator. Additionally, $\{\varphi_i\}_{i \in \mathbb{N}}$ is a set of basis shape functions and $q(t)$ are modal coefficients. These modal coefficients can be thought of as generalized coordinates of the finite-dimensional system. \add{Several shape functions exist that can be used to approximate the solution, such as chebyshev polynomials and legendre polynomials. Each degree of shape-function gives the model a certain amount of flexibility. Therefore increasing the order of shape-functions allows the model to describe more complex robot configurations.}\todo{add more details about shapefunctions} It is important the shape functions are orthogonal which means that $\int_\mathbb{X} \varphi_i \varphi_j d \sigma = 0$ for any $i \neq j$ and non-zero otherwise. Given this, the $n$-th order strain expansion can be formulated as,

%% maybe write about what shape functions were used, 

\begin{equation}
\begin{aligned}
    \xi(\sigma,t) \cong & \hspace{5pt}  (B_a \otimes [ \varphi_1 \dots \varphi_N ])q(t)\\ = &  \underbrace{ \begin{pmatrix}
    \varphi_1(\sigma) & \dots  & \varphi_N(\sigma) & \dots     & 0      & \dots  &  0 \\
    \vdots    & \ddots & \vdots    & \ddots    & \vdots & \ddots & \vdots \\
    0         & \dots  & 0         & \dots     & \varphi_1(\sigma) & \dots & \varphi_N (\sigma)
    \end{pmatrix}}_{\Phi(\sigma)} \begin{pmatrix} q_1(t) \\ \vdots \\ q_n(t) \end{pmatrix} +  \begin{pmatrix} \xi_{1,0} \\ \vdots \\ \xi_{n,0}   \end{pmatrix}
    \end{aligned},
\label{eq2:xishape}
\end{equation}

where $\Phi : \mathbb{R} \mapsto \mathbb{R}^{m \times n}$ is a shape function matrix in which $n$ is equal to the amount of modal coordinates, and $m$ the amount of active strains, and $B_a \subseteq \text{span}(\mathbb{I}_6)$ a selection matrix of unconstrained strains. Within matrix $B_a$ the active strains are given a `1' and constrained stains `0'. For the planar soft robot one curvature-torsion strain and one stretch-shear strain are active. 


In Matlab \cite{MATLAB2020} a code was created to calculate the forward kinematics given modal coordinates $q(t)$. To reduce computation time of solving the ODE with \verb+ode45+, the 4 by 4 matrix $g(\sigma,t)$, Equation (\ref{eq2:g}), is rewritten as a column vector of size 7. To this end, the rotation matrix $R(\sigma,t)$ is rewritten to quaternion formulation as follows \cite{Boyer2019},


\begin{equation}
\frac{\partial}{\partial \sigma}    \begin{pmatrix} Q \\ p \end{pmatrix} = \begin{pmatrix} 2 ||Q||^{-1} A(R(Q)K)Q \\ R(Q)p \end{pmatrix}
\label{eq2:Qp}
\end{equation}

where $Q$ is the quaternion representation of rotation matrix $R$, and $R(Q)$ a function mapping a quaternion to rotation matrix representation, $A$ is a function defined as,


\begin{equation}
    A(K) = \begin{pmatrix} 0 & -K_1 & -K_2 & -K_3 \\ K_1 & 0 & -K_3 & \hspace{8pt}K_2 \\ K_2 & \hspace{8pt}K_3 & 0 & -K_1 \\ K_3 & -K_2 & \hspace{8pt}K_1 & 0 \end{pmatrix}
    \label{eq2:AK}
\end{equation}

where $K_i$ correspond to the entries of $\xi(\sigma,t)$ shown in Equation (\ref{eq2:etaxi}). In Equation \ref{eq2:Qp} and \ref{eq2:AK} the dependency of $\sigma$ and $t$ has been omitted for the sake of readability.

\subsection{Inverse kinematics}

The developed forward kinematic model can be used for determining the inverse kinematics of the planar robot. This is done by a Jacobian search method. In essence, a desired position and orientation of the robot's end-effector is given. The Jacobian inverse technique is used to find the modal coordinates $q(t)$ mapping to desired end-effector position $x_d = [x,y]^\top$. For the forward kinematics the following can be stated,

\begin{equation}
    x = f(q)
\end{equation}

where, $x \in \mathbb{R}^n$ is the position and orientation vector of the end-effector. For the planar robot $n = 3$, as two translations and one coupled rotation can be achieved. Modal coordinates are stored in vector $q \in \mathbb{R}^m$, which can take any size. Lastly, $f$ a function which describes the forward kinematics performing $x \mapsto q$. The inverse kinematic problem aims to find,

\begin{equation}
    q = f^{-1}(x)
    \label{eq2:q}
\end{equation}

The modal coordinates corresponding to the desired end-effector position and orientation can be found by using the inverse jacobian method, with the updating scheme as,

\begin{equation}
    q_{k+1} = q_k + \alpha J^\dagger \beta e_k \hspace{20pt} \text{with} \hspace{10pt}  e_k = x_d - x_k
    \label{eq2:qupdate}
\end{equation}

where $q_k$ are the modal coordinates at iteration $k$, $\alpha$ a learning gain used to enhance fast convergence $\beta$ is a diagonal matrix used to prioritize desired end-effector coordinates. In our case, the position in the $x-y$ plane is more important than the orientation $\theta$. Lastly, $J^\dagger$ is an adapted form of the damped Moore–Penrose pseudo inverse defined as,

\begin{equation}
    J^\dagger = wJ^\top(JwJ^\top + \rho I)^{-1}
    \label{eq2:pseudoinverse}
\end{equation}

In this equation $w$ is a diagonal weighing matrix used to prioritize modal coordinates of which the strain is approximated with lower order shape functions, this is done to reduce the dimension of the system as far as possible. The factor $\rho I \in \mathbb{R}^{6 \times 6}$ is introduced to avoid singularities of the jacobian matrix $J$. The Jacobian at each increment is determined with the following equation \cite{Caasenbrood2020},

\begin{equation}
\eta(\sigma,t) = Ad_{g^{-1}(\sigma = l)} \int_{0}^{\sigma} Ad_g B_a \Phi(\sigma) d\sigma \dot{q} := J\dot{q}
\label{eq2:J}
\end{equation}

where $J(\sigma,q) \in \mathbb{R}^{6 \times n}$ is the jacobian matrix mapping modal velocities to the time-twist field, and $Ad_g$ is defined as the adjoint mapping found by,


\begin{equation}
    Ad_g = \begin{pmatrix} R & 0 \\ \hat{r}R & R \end{pmatrix}
    \label{eq2:Adg}
\end{equation}

in which, $R$ is the rotation matrix at space instant $\sigma$, and $\hat{r}$ a 3x3 skew-symmetric matrix for which holds that $\hat{r}V = r \times V$ for any $V \in \mathbb{R}^3$ \cite{Boyer2019}.

An algorithm capable of finding a inverse kinematic solution is created based on Equation (\ref{eq2:q}) - (\ref{eq2:Adg}). The algorithm is initiated with a desired end-effector position $x_d$ and guess for modal coordinates $q_0$. Furthermore, the amount of shape functions used to approximate the end-effector position is user defined. This results in
\todo{ cite source on pseudo inverse and inverse jacobian technique},


\begin{algorithm}[H]
\caption{Numerical Inverse Kinematics}
\begin{algorithmic}[1]
\State $q$ $\leftarrow$ $q_0$ \Comment{Initial condition}
\While{$|| x_d - x_k || > \epsilon$}{}      \Comment{While error is larger than some $\epsilon$}
    \State $x_k$ $\leftarrow$ $f(q_k)$  \Comment{Position at current generalized coordinate}
     \State $e_k$ $\leftarrow$ $x_k - x_d$ \Comment{Current error}
        \For{$0:\Delta \sigma:L$}
            \State $J_{\sigma}$ $\leftarrow$  $J_{\sigma}$ + $Ad_{g(\Delta \sigma)} B_a \Phi(\Delta \sigma)$ \Comment{Numerical integration over length L}
        \EndFor
    \State \textbf{end}
    \State $J$ $\leftarrow$ $Ad_{g^{-1},\sigma=L}$ $J_{\sigma}$ \Comment{Compute Jacobian}
    \State $J^{\dagger}$ $\leftarrow$  $w$ $J^{\top}$ $(J^\top w J - \rho \mathbb{I}_6)^{-1}$ \Comment{Calculate pseudo inverse}
    \State $q_{k+1}$  $\leftarrow$ $q_{k}$ + $\alpha$ $J^{\dagger} \beta e_k$ \Comment{Update generalized coordinates}
\EndWhile 
\State \textbf{end}
    \label{alg2:numericalinverse}
\end{algorithmic}
\end{algorithm}



