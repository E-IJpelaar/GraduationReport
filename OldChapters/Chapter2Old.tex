

\section{Forward kinematics}

The forward kinematics of the robot manipulator are derived by using the model formulated in \cite{Caasenbrood2020}. Here, the soft robotic manipulator is regarded as a continuous deformable structure with infinite degrees of freedom. A one-dimensional Cosserat beam model is used to describe the configuration space of the robot. This beam model can be thought of as a continuous curve representing the robot's backbone. This curve describes the configuration of the soft robot as a function of space and time. To this end, spatial coordinate $\sigma \in \mathbb{X}$ within bounded domain $\mathbb{X} \in [0,l] \subset \mathbb{R}$ is introduced. Furthermore, a temporal coordinate $t \in  \mathbb{T}$ with $\mathbb{T} \subseteq \mathbb{R}$ is defined. This allows to describe position $p(\sigma,t) \in \mathbb{R}^3$ and rotation $R(\sigma,t) \in \mathbb{SO}(3)$ 
for any point $\sigma$ and time instance $t$ among the backbone of the soft manipulator by,


\begin{equation}
    g(\sigma,t) := \begin{pmatrix}  R(\sigma,t) & p(\sigma,t) \\ 0_3^\top & 1 \end{pmatrix} \in \mathbb{SE}(3),
    \label{eq2:g}
\end{equation}

where $\mathbb{SE}(3)$ is the Lie group of rigid body transformations in $\mathbb{R}^3$ \cite{Sola2018}. The partial derivatives of $g$ can be described by two vector fields in the Lie algebra $\mathfrak{se}(3)$. Here $\partial (.)/\partial(\sigma)$ and $\partial (.)/\partial(t)$ will be denoted by a `prime' and `dot' for position and time, respectively. The partial derivative with respect to time of (\ref{eq2:g}) results in the time-twist field as described by,

\begin{equation}
    \frac{\partial g}{\partial t} = \dot{g} = g\hat{\eta} \Longrightarrow \hat{\eta} := g^{-1}\dot{g} = \begin{pmatrix} \Omega_\times & V \\ 0_3^\top & 0 \end{pmatrix} \in  \mathfrak{se}(3),
    \label{eq2:dotg}
\end{equation}

where $\Omega_\times$ is a skew-symmetric matrix expressing angular velocity, and vector $V = (v_1,v_2,v_3)^\top $ contains linear velocities. From this skew-symmetric matrix, vector $\Omega = (\omega_1,\omega_2,\omega_3)^\top$ can be derived \cite{Sola2018}. Since $\mathfrak{se}(3) \cong \mathbb{R}^6$, (\ref{eq2:dotg}) can be reformulated as,

\begin{equation}
    \eta(\sigma,t) = \begin{pmatrix} \Omega \\ V \end{pmatrix} \in \mathbb{R}^6,
\end{equation}

where vector field $\eta(\sigma,t)$ describes an infinitesimal local transformation by a frame at position $\sigma$ between infinitesimally small $t$ and $t+\Delta t$. Following the same logic, the partial derivative of (\ref{eq2:g}) with respect to space results in the space-twist field defined as,

\begin{equation}
   \frac{\partial g}{\partial \sigma} =  g' = g\hat{\xi} \Longrightarrow \hat{\xi} := g^{-1}g' = \begin{pmatrix} K_\times & E \\ 0_3^\top & 0 \end{pmatrix} \in  \mathfrak{se}(3),
    \label{eq2:gaccent}
\end{equation}

in which $K_\times$ is a skew-symmetric matrix expressing curvature-torsion strain, and $E = (\epsilon_1,\epsilon_2,\epsilon_3)^\top$ a vector containing stretch-shear strain. Again,  $K = (k_1,k_2,k_3)^\top$ follows from $K_\times$. Likewise, (\ref{eq2:gaccent}) can be formulated as,


\begin{equation}
    \xi(\sigma,t) = \begin{pmatrix} K \\ E \end{pmatrix} \in \mathbb{R}^6,
    \label{eq2:etaxi}
\end{equation}

where vector field $\xi(\sigma,t)$ describes an infinitesimal local transformation by a frame at time instance $t$ between infinitesimally small $\sigma$ and $\sigma+\Delta \sigma$.



For solving the forward kinematics only position of the soft robotic manipulator is of interest. Hence, first order partial differential equation (PDE)

\begin{equation}
    \frac{\partial g}{\partial \sigma} = g \hat{\xi},
    \label{eq2:dgdsigma}
\end{equation}

needs to be solved. To solve this PDE, it is transformed to an ordinary differential equation (ODE) by exploiting the Galerkin reduction method 
\cite{Galerkin}. Here, the infinite dimensional system is projected onto a subspace of finite dimension that contains basis elements of the expected solution. By reducing the dimensionality of the system, higher order dynamics are not captured in the model and thus robustness should be taken into account. To transform PDE (\ref{eq2:dgdsigma}), the components of the strain field $\xi := (g^{-1}g')$ are approximated using a finite amount of shape functions as,

\begin{equation}
    \xi_i(\sigma,t) \approx \sum_{i=1}^N \varphi_i(\sigma)q_i(t) + \xi_{i,0}(\sigma), \hspace{20pt} \forall \sigma \in \mathbb{X}, t \in \mathbb{T},
\end{equation}

in which, $\xi_{i,0}$ corresponds to the undeformed configuration of the robotic manipulator. Additionally, $\{\varphi_i\}_{i \in \mathbb{N}}$ is a set of basis shape functions and $q(t)$ are modal coefficients. These modal coefficients can be thought of as generalized coordinates of the finite-dimensional system. To be clear on this reduction method, integer $N$ is the amount of shape functions used to estimate strain field $\xi(\sigma,t)$. Hence, index $i$ represents the order of the shape function. Several shape functions exist that can be used to approximate the solution. Chebyshev polynomials and Legendre polynomials are a few of those \cite{Galerkin}. Each degree of shape-function gives the model a certain amount of flexibility. Therefore increasing the order of shape-functions allows the model to describe more complex robot configurations. To avoid coupling between shape functions,  it is important that these shape functions are orthogonal. This means that $\int_\mathbb{X} \varphi_i \varphi_j d \sigma = 0$ for any $i \neq j$ and non-zero otherwise. Given this, the $n$-th order strain expansion can be formulated as,

%% maybe write about what shape functions were used, 

\begin{equation}
\begin{aligned}
    \xi(\sigma,t) \cong & \hspace{5pt}  (B_a \otimes [ \varphi_1 \dots \varphi_N ])q(t)\\ = &  \underbrace{ \begin{pmatrix}
    \varphi_1(\sigma) & \dots  & \varphi_N(\sigma) & \dots     & 0      & \dots  &  0 \\
    \vdots    & \ddots & \vdots    & \ddots    & \vdots & \ddots & \vdots \\
    0         & \dots  & 0         & \dots     & \varphi_1(\sigma) & \dots & \varphi_N (\sigma)
    \end{pmatrix}}_{\Phi(\sigma)} \begin{pmatrix} q_1(t) \\ \vdots \\ q_n(t) \end{pmatrix} +  \begin{pmatrix} \xi_{1,0} \\ \vdots \\ \xi_{n,0}   \end{pmatrix}
    \end{aligned},
\label{eq2:xishape}
\end{equation}

where $\Phi \in \mathbb{R}^{m \times n}$ is a shape function matrix in which $n$ is equal to the amount of modal coordinates, and $m$ the amount of active strains, and $B_a \subseteq \text{span}(\mathbb{I}_6)$ a selection matrix of unconstrained strains. Within matrix $B_a$ the active strains are given a `1' and constrained stains `0'. For the planar soft robot one curvature-torsion strain and one stretch-shear strain are active, implying that $m = 2$. The amount of shape functions used to approximate a strain affects the value of $n$. Approximating, the two strains with a single shape functions results in $n = 2$. Hence, $n$ can be found with the simple relation: amount of shape functions used in approximation multiplied by $m$.


In Matlab \cite{MATLAB2020} a code was created to calculate the forward kinematics given modal coordinates $q(t)$. To reduce computation time, the rotation matrix in $g(\sigma,t)$ (\ref{eq2:g}) was reformulated using quaternion formulation \cite{Boyer2019}. This allows to express any rotation by a vector of length 4, instead of 3 by 3 matrix. Therefore $g(\sigma,t)$ can significantly be reduced from 16 entries, to only 7 entries when using quaternion formulation. Any rotation matrix $R(\sigma,t)$ can be rewritten to quaternions by,


\begin{equation}
\frac{\partial}{\partial \sigma}    \begin{pmatrix} Q \\ p \end{pmatrix} = \begin{pmatrix} 2 ||Q||^{-1} A(R(Q)K)Q \\ R(Q)p \end{pmatrix},
\label{eq2:Qp}
\end{equation}

where $Q \in \mathbb{R}^4$ is the quaternion representation of rotation matrix $R$, and $R(Q)$ a function mapping a quaternion to rotation matrix representation, $A$ is a function defined as,


\begin{equation}
    A(K) = \begin{pmatrix} 0 & -K_1 & -K_2 & -K_3 \\ K_1 & 0 & -K_3 & \hspace{8pt}K_2 \\ K_2 & \hspace{8pt}K_3 & 0 & -K_1 \\ K_3 & -K_2 & \hspace{8pt}K_1 & 0 \end{pmatrix},
    \label{eq2:AK}
\end{equation}

where $K_i$ correspond to the entries of $\xi(\sigma,t)$ shown in (\ref{eq2:etaxi}). In  (\ref{eq2:Qp}) and (\ref{eq2:AK}) the dependency of $\sigma$ and $t$ has been omitted for the sake of readability.



