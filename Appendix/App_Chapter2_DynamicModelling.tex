\section{Inverse kinematics}
\label{app:chap2}

%This subsection is not definite. 2 inverse kinematic scripts have been created. One in which only end-effector position $[x,y]$ is given. One where also the orientation is added $[x,y,\theta]$ . For only a single shape-function approximation (constant curvature approach), will find the shortest path. However, when increasing the amount of shape functions, both models will not come with the most shortest path. Therefore, the inverse kinematics as described here has not been used extensively.


Inverse kinematics allows to map the desired end-effector position to the corresponding modal coordinates. To determine this inverse mapping the Jacobian search method is exploited \cite{JacobianInverse}. Essentially, a desired position and orientation of the robot's end-effector is given. 


The derived forward kinematics can be regarded as,

\begin{equation}
    r = f(q),
\end{equation}

where function $f(q)$ is used to calculate the end-effector's position and orientation based on modal coordinate vector $q$. The position and orientation of the end-effector are given by vector $r$. The inverse kinematic problem aims to find a mapping $q\mapsto r$. This mapping is obtained by exploiting the inverse jacobian method according to an iterative updating scheme as,

\begin{equation}
    q_{k+1} = q_k + \chi [J(\sigma)]_1^\dagger Q e_k \hspace{20pt} \text{with} \hspace{10pt}  e_k = x_d - x_k,
    \label{eq2:qupdate}
\end{equation}

where $q_k$ is modal coordinate vector at iteration $k$, $\chi \in \mathbb{R}^+$ a learning gain to enhance fast convergence. Diagonal weighing matrix $Q \in \mathbb{R}^{3\times 3}$ can be used to prioritize position or orientation. In most cases, position is preferred over (the coupled) orientation. Lastly, $J^\dagger$ is an adapted form of the damped Moore–Penrose pseudo inverse and defined as,

\begin{equation}
    J^\dagger = WJ^\top(JWJ^\top + \mu^2 I)^{-1}
    \label{eq2:pseudoinverse},
\end{equation}

where $W \in \mathbb{R}^{2\times 2}$ is a diagonal weighing matrix and factor $\mu^2 I \in \mathbb{R}^{3 \times 3}$ is introduced to avoid singularities of the jacobian matrix $J \in \mathbb{R}^{3 \times 2}$. 

An algorithm capable of finding a inverse kinematic solution is created based on Equation (\ref{eq2:qupdate}). The algorithm is initiated with a desired end-effector position and orientation $r_{set}$ and initial guess for modal coordinates $q_0$. This results in,


\begin{algorithm}[H]
\caption{Numerical Inverse Kinematics}
\begin{algorithmic}[1]
\State $q$ $\leftarrow$ $q_0$ \Comment{Initial condition}
\While{$|| r_{set} - r_k || > \delta$}{}      \Comment{While error is larger than some $\delta$}
    \State $r_k$ $\leftarrow$ $f(q_k)$  \Comment{Position at current generalized coordinate}
     \State $e_k$ $\leftarrow$ $r_{set} - r_{k}$ \Comment{Current error}
        \For{$0:\Delta \sigma:L$}
            \State $J_{\sigma}$ $\leftarrow$  $J_{\sigma}$ + $Ad_{g(\Delta \sigma)} B_a \Phi(\Delta \sigma)$ \Comment{Numerical integration over length L}
        \EndFor
    \State \textbf{end}
    \State $J$ $\leftarrow$ $Ad_{g^{-1},\sigma=L}$ $J_{\sigma}$ \Comment{Compute Jacobian}
    \State $J^{\dagger}$ $\leftarrow$  $w$ $J^{\top}$ $(J^\top w J - \mu^2 \mathbb{I}_3)^{-1}$ \Comment{Calculate pseudo inverse}
    \State $q_{k+1}$  $\leftarrow$ $q_{k}$ + $\chi$ $J^{\dagger} Q e_k$ \Comment{Update generalized coordinates}
\EndWhile 
\State \textbf{end}
    \label{alg2:numericalinverse}
\end{algorithmic}
\end{algorithm}


\section{Velocity kinematics}


Using the equality of mixed partial derivatives, at each instant of space and time $\frac{\partial}{\partial t}g' = \frac{\partial}{\partial \sigma}\dot{g}$ holds \cite{Caasenbrood2020}. Substituting (\ref{eq2:dgdsigma}) and (\ref{eq2:dgdt}) into this relation and simplifying results in \cite{Caasenbrood2020},

\begin{equation}
    \hat{\eta}' = -(\hat{\xi}\hat{\eta} - \hat{\eta}\hat{\xi}) + \Dot{\xi},
        \label{eq2:pde2}
\end{equation}

where the Lie bracket $\xi$ and $\eta$ can be recognized between the parenthesis. The Lie bracket $[\hat{\xi},\hat{\eta}]$ also belongs to the group of Lie algebra $\mathfrak{se}(2)$. Therefore, it can be expressed by the adjoint action between $\xi$ onto $\eta$. This adjoint action maps the PDE of (\ref{eq2:pde2}) from $\mathbb{R}^{3\times 3}$ to $\mathbb{R}^3$. This allows to write the velocity kinematics in vector notation as \cite{Caasenbrood2020},

\begin{equation}
    \eta'= -\text{ad}_\xi \eta + \Dot{\xi},
    \label{eq2:etapde}
\end{equation}

where $\text{ad}_\xi$ denotes the adjoint action of the algebra $\hat{xi} \in \mathfrak{se}(2)$. Exploiting the relation $\frac{d}{d \sigma} \text{Ad}_g = \text{Ad}_g \text{ad}_{g^{-1}g'}$ \cite{Boyer2019}, \cite{traversaro2016multibody}, with $g^{-1}g' = \xi$ which follows from (\ref{eq2:dgdsigma}), it follows that $-\text{ad}_\xi = (\text{Ad}_g^{-1})'\text{Ad}_g$. Substitution of this expression into (\ref{eq2:etapde}) allows to formulate the spatial derivative of the velocity twist as,

\begin{equation}
    \eta'= (\text{Ad}_g^{-1})'\text{Ad}_g \eta + \Dot{\xi},
    \label{eq2app:etadif}
\end{equation}

where $\text{Ad}_g \in \mathbb{R}^{3 \times 3}$ is the adjoint mapping of $g$ \cite{Sola2018}. An analytic solution can be obtained by integrating (\ref{eq2app:etadif}) over spatial domain $[0,\sigma]$. 