In the past decades, the field of soft robotics has been receiving a substantial amount of research interest. In contrast to their rigid counterparts, soft robots distinguish themselves as intrinsically dexterous, flexible and resilient. These inherent properties can reshape our perspective on robotics. Their potential is readily acknowledged in the fields of medical sciences and co-robotics. 



In this work, soft robots applied for object manipulation are considered. Recent advancements have led to a novel technique of modelling soft robots by exploiting partial differential equations. This method allows to accurately describe, the possible complex, configurations employing a continuous curve. The theoretical infinite degree-of-freedom soft robotic system is reduced to a finite-dimensional model equivalent by exploiting the Galerkin reduction method. This allows the application of well-established theories, including Lie group theory, to derive a dynamic model. In this study, the proven method of Jacobian transposed control is transferred to the domain of soft robotics. The performance of this feedback controller is evaluated for simulated dynamic responses and experiments conducted on a physical setup. The achieved results demonstrate that Jacobian control can be considered a valuable alternative over more advanced model-based control methods.